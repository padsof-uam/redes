\documentclass{apuntes}

\usepackage{tikz}
\usepackage{tikz-3dplot}
\title{Memoria de la práctica 3}
\author{Guillermo Julián y Víctor de Juan}
\date{20/11/2013}
\usetikzlibrary{arrows,calc,shapes}

\begin{document}

\pagestyle{plain}
\maketitle
\tableofcontents

\chapter{Porcentajes de paquetes}


Al analizar los porcentajes de paquetes descartados y de los paquetes que no son ni TCP ni UDP nos encontramos con la existencia de algunos paquetes que tenían de protocolo $0x8100$ en vez de $0x8060$ que sí eran UDP. Casualmente estos paquetes eran todos los que pensábamos que eran UDP. Tras corregir esto, las estadísticas (de la traza practica3\_rc1lab.pcap.) son:

\easyimg{imgs/Stats.png}{Porcentajes de paquetes leidos}{lblStats}

En donde el tiempo está sacado de las cabeceras de los paquetes, para hacer una estadística más real.

\newpage
\section{Top}

\subsection{Top de IP's}

A continuación mostramos las IP's más usadas separadas por origen (recibidos) o destino (enviados) y separadas también por bytes o paquetes.

\easyimg{imgs/top5_IP.png}{Top de las 5 IP's más activas}{lblTop5IP}
\newpage
\subsection{Top de puertos}

A continuación mostramos los puertos más usados separados por origen (recibidos) o destino (enviados) y separados también por bytes o paquetes.

\easyimg{imgs/Top5_puertos.png}{Top de los 5 puertos más activos}{lblTop5Puertos}

\chapter{Gr\'aficas}

\section{ECDF de los tamaños de las trazas}

\section{Caudal de nivel 2}

\section{Tiempos de llegadas}


\end{document}